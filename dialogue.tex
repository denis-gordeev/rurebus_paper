\documentclass{dialogue}

%\usepackage{bookta}sb

\begin{document}

\begin{otherlanguage}{english}
\begin{center}
{\Large\bfseries{Joint task learning for relation extraction and named entity recognition}}

\medskip

%Adis Davletov (\texttt{davletov-aa@ranepa.ru}), Denis Gordeev (\texttt{gordeev-di@ranepa.ru}), Alexey Rey (\texttt{rey-ai@ranepa.ru})
Authors

\medskip

%RANEPA, Moscow, Russia
Institution
\end{center}

In this work we present our system for RuREBus challenge held together with Dialog 2020 conference. The task consisted of 3 tracks: named entity recognition, relation extraction with provided named entity tags and end-to-end relation extraction. Our system took the first place in the named entity recognition track and the second place in the third track. For the second task we failed to submit the solution till the deadline but it was among the best systems. The systems for all tasks are based on Transformer models.

\textbf{Key words:} relation extraction, named entity recognition, transformer, bert
\end{otherlanguage}

\bigskip

\begin{otherlanguage}{russian}
\begin{center}
{\Large\bfseries{Совместное обучение моделей для извлечения отношений и именованных сущностей}}

\medskip

%Давлетов А. А. (\texttt{davletov-aa@ranepa.ru}), Гордеев Д. И. (\texttt{gordeev-di@ranepa.ru}), \\Рей А. И. (\texttt{rey-ai@ranepa.ru})
Авторы

\medskip

%РАНХиГС, Москва, Россия
Организация
\end{center}

В данной работе мы представляем нашу систему для соревнования RuREBus, проводящегося совместно с конференцией Dialog 2020. Задача состояла из 3 дорожек: распознавание именованных сущностей, классификация отношений между заранее аннотированными именованными сущностями и извлечение отношений из неаннотированного текста. Наша система заняла первое место в задаче распознавания именованных сущностей и второе место на третьей дорожке. Для второй задачи мы не успели своевременно представить решение, но оно оказалось в числе лучших систем. Системы для всех задач основаны на моделях Transformer.
\medskip

\textbf{Ключевые слова:} извлечение отношений, распознавание именованных сущностей, transformer, bert
\end{otherlanguage}

\selectlanguage{english}

\section{Introduction}
There are many ways to extract information from text. One of the most popular approaches is to extract named entities and classify relations between them. One of the most popular datasets for this task is TACRED \cite{tacred} where semantic relations are understood as relations between two pairs of entities. 

<TACRED METHODS>

Unfortunately, such annotated datasets are scarce for most languages besides English. Some researchers have tried to solve this problem for the Russian language. They have used unsupervised approaches based on knowledge databases such as Wikidata and online encyclopedias such as Wikipedia. Models trained this way tend to be not specialized because the original database does not contain relations from the required domain. They also tend to work only for the most popular relation types such as geographical or professional ones which are common to Wikipedia.

There are few annotated datasets for the Russian language. Among similar tasks to relation extraction there was held FactRuEval 2016 within the conference Dialog 2016. Within the competition contestants had to extract facts from news articles and to fill special slots in these facts (e.g. one of the fact types was 'Occupation' and its fields were 'POSITION', 'WHO', 'WHERE' and 'PHASE').

RuREBus competition was devoted to the problem of relation extraction and named entities recognition in a specialized business domain.

\section{Shared task overview}
The organizers of the competition have provided 188 annotated texts as the training dataset and 544 texts as the test dataset for the first and thirds tracks and <N> <N> for the second track respectively. All texts were provided by the Ministry of Economic Development of the Russian Federation. The corpus consists of various regional and strategic plan reports. There are in total 8 named entity classes and 11 semantic relation classes. The organizers have also provided a large unannotated dataset for language model fine-tuning. However, we did not use it.

TEXT EXAMPLE
 
Named entity groups could contain rather broad types of entities, for example "SOC" entities contained social groups as well as various social attributes - phrases like 'blue collar workers' and 'housing accessibility' corresponded to this group.
\subsection{Dataset}
\begin{table}[bth]
	\centering
	\small
	\begin{tabular}{c||p{8cm}}
		\hline
		Type & Description\\ \hline
		MET & Some quantitative metric \\ \hline
		ECO & An economy entity or facility\\ \hline
		BIN & A binary attribute\\ \hline
		CMP & Comparative attribute\\ \hline
		QUA & Qualitative attribute\\ \hline
		ACT & Activity, actions, implemented policies\\ \hline
		INST & Institutions and organizations\\ \hline
		SOC & Social groups and characteristics\\ \hline
	\end{tabular}
	\caption{Named entity types}
	\label{tab:ner}
\end{table}
\begin{table}[bth]
	\centering
	\small
	\begin{tabular}{c|c|p{8cm}}
		\hline
		Group & Type & Description\\ \hline
		Current state of affairs & NNG & now negative \\ \hline
		Current state of affairs & NNT & now neutral \\ \hline
		Current state of affairs & NPS & now positive \\ \hline\hline

		Results & PNG & past negative\\ \hline
		Results & PNT & past neutral\\ \hline
		Results & PNS & past positive\\ \hline\hline

		Forecasts & FNG & future negative\\ \hline
		Forecasts & FNT & future neutral\\ \hline
		Forecasts & FNS & future positive\\ \hline\hline

		Goals & GOL & some abstract goals\\ \hline
		Tasks & TSK & tasks and performed actions to achieve goals\\ \hline

	\end{tabular}
	\caption{Semantic relation types}
	\label{tab:rel}
\end{table}
\section{Our solution}
\subsection{Named entities recognition}
\subsection{Stand-alone relation extraction}
\subsection{Relation classification with provided named entity tags}
\section{Results}
\section{Conclusion}
\bibliography{semeval2018}


\end{document}
